\documentclass[11pt]{article}
\documentclass[11pt]{article}
\documentclass{article}
\usepackage{libertine}
\usepackage{amssymb}
\usepackage{amsmath}
\usepackage[makeroom]{cancel}

% To produce a letter size output. Otherwise will be A4 size.
\usepackage[letterpaper]{geometry}
% To produce a letter size output. Otherwise will be A4 size.
\usepackage[letterpaper]{geometry}

% For enumerated lists using letters: a. b. etc.
\usepackage{enumitem}

\topmargin -.5in
\textheight 9in
\oddsidemargin -.25in
\evensidemargin -.25in
\textwidth 7in
\usepackage{pifont}

\begin{document}





\begin{center}
    {\Large Valentinno Cruz\\
    Homework Assignment 4\\
    March. 31st, 2021\\}

\end{center}

%-----------        ----------- 
%----------- Part 1 ----------- 
%-----------        ----------- 


\begin{flushleft}
{\Large 1. Functions}
\end{flushleft}




\begin{enumerate}

\begin{flushleft}
{\large 1. Is $\mathnormal{f}$ a function from $\mathbb{ R}$ 
to $\mathbb{ R}$ if}\\
\end{flushleft}



%========= Problem A .===========


\large (a) $\mathnormal{f}$(x) = \textpm \sqrt{(x^2+1)}?\\

\begin{itemize}
\item \textbf{Solution}\\
\large This is not a function because it is not \\
\large uniquely assigned to the point $x \in  \mathbb{ R}$ \\

\end {itemize}
\end {enumerate}


%========= Problem B .===========
\begin{enumerate}

\begin{flushleft}
\end{flushleft}

\large (b) $\mathnormal{f}$(x) = 1/$x$\\

\begin{itemize}
\item \textbf{Solution}\\
\large This is not a function from $\mathbb{ R}$ to $\mathbb{ R}$ \\
\large because for the element $x$ = 0, there is no image \\

\end {itemize}
\end {enumerate}




%========= Problem C .===========
\begin{enumerate}

\begin{flushleft}
\end{flushleft}

\large (c) $\mathnormal{f}$(x) = $x-x^2$\\

\begin{itemize}
\item \textbf{Solution}\\
\large This is not a function from $\mathbb{ R}$ to $\mathbb{ R}$ \\
\large because $x=0$ and $x=1$ are both mapped to element 0 \\
\large $\therefore $ it's condition is not unique\\
\end {itemize}
\end {enumerate}



\pagebreak


%-----------        ----------- 
%----------- Part 2 ----------- 
%-----------        ----------- 





\begin{enumerate}

\begin{flushleft}
{\large 2. Find the domain and range of these functions. Note that in each case, to find the domain, determine
the set of elements assigned values by the function.}\\
\end{flushleft}


%========= Problem A .===========

\large (a) The function that assigns to each positive integer the largest perfect square not exceeding this
integer.\\

\begin{itemize}
\item \textbf{Solution}\\
\large Domain = $\mathbb{N}$, all positive integers.\\
\large Range = set of all perfect squares.\\
\large - $\mathnormal{f}$ is not one to one because
if we take \\
\large - $\mathnormal{f}$(2)=2 and$\mathnormal{f}$(3)=2,
 the largest perfect sq is $\leq$ 3\\
 \large - so we can say the elements have the same image\\
 \large $\therefore$ $\mathnormal{f}$ is onto\\

\end {itemize}
\end {enumerate}


%========= Problem b .===========
\begin{enumerate}

\begin{flushleft}
\end{flushleft}


\large (b) The function that assigns to each bit string the number of ones in the string minus the number
of zeros in the string.\\

\begin{itemize}
\item \textbf{Solution}\\
\large $\mathnormal{f}$(x) = (number of 1's in string) - (number of 0's in string)\\
\large ex. $\mathnormal{f}$(1100) = 2-2 = 0\\
\large Domain = set of all strings\\
\large Range = set of all integers ($\mathbb{Z})$\\
\large - $\mathnormal{f}$ is not one to one because
if we take  $\mathnormal{f}$(1100) = 0 and $\mathnormal{f}$(10) = 0\\
\large - Different elements give us the same image\\
\large $\therefore$ $\mathnormal{f}$ is onto\\

\end {itemize}
\end {enumerate}


%========= Problem c .===========
\begin{enumerate}

\begin{flushleft}
\end{flushleft}


\large (c) The function that assigns to each bit string twice the number of zeros in that string.\\

\begin{itemize}
\item \textbf{Solution}\\
\large $\mathnormal{f}$(x) = 2*number of zeros\\
\large ex. $\mathnormal{f}$(100) = 2*2=4\\
\large Domain = set of all binary strings\\
\large Range = $\{2*n \colon n\in \{1,2, \dots \}\}$\\
\large .\hspace{1.1cm}= set of all even numbers\\
\large - $\mathnormal{f}$ is not one to one because
if we take \\
\large $\mathnormal{f}$(001) = 2*2 = 4 and 
$\mathnormal{f}$(100) = 2*2 = 4\\
\large - we get the same image with different elements\\
\large $\therefore$ $\mathnormal{f}$ is onto\\

\end {itemize}
\end {enumerate}







%========= Problem d .===========
\begin{enumerate}

\begin{flushleft}
\end{flushleft}


\large (d) The function that assigns the number of bits left over when a bit string is split into bytes (which
are blocks of 8 bits).\\

\begin{itemize}
\item \textbf{Solution}\\
\large $\mathnormal{f}$(x) = the num of bits of a string
left over when a bit string is split into bytes\\
\large ex. $\mathnormal{f}$(100001101) = 0\\
\large $\mathnormal{f}$(0000000000) = 2\\
\large Domain = set of all binary strings\\
\large Range = set of all non-negative integers\\
\large .\hspace{1.1cm}= \{0,1,2,3, \dots \}\\
\large - $\mathnormal{f}$ is not one to one because
if we take \\
\large - $\mathnormal{f}$(01) = 2 and 
$\mathnormal{f}$(10) = 2\\
\large - we get the same image with different elements\\
\large $\therefore$ $\mathnormal{f}$ is onto\\

\end {itemize}
\end {enumerate}

%========= Problem e .===========
\begin{enumerate}

\begin{flushleft}
\end{flushleft}


\large (e) The function that assigns to a bit string the number of one bits in the string.
For example, the domain for the question is the set of bit strings (of any length). (Remember, a
bit string is a string made up of just 0’s and 1’s.) The range is the set of nonnegative integers:
{0, 1, 2, ...}. That is, the number of 1’s in a bit string can be 0, 1, 2, ....\\

\begin{itemize}
\item \textbf{Solution}\\
\large $\mathnormal{f}$(x) = the num 1's in $x$\\
\large ex. $\mathnormal{f}$(00000) = 0\\
\large .\hspace{.5cm}$\mathnormal{f}$(110) = 2\\
\large Domain = set of all binary strings\\
\large Range = set of all non-negative integers\\
\large .\hspace{1.1cm}= \{0,1,2,3, \dots \}\\
\large - $\mathnormal{f}$ is not one to one because
if we take \\
\large - $\mathnormal{f}$(01) = 1 and 
$\mathnormal{f}$(10) = 1\\
\large - we get the same image with different elements\\
\large $\therefore$ $\mathnormal{f}$ is onto\\

\end {itemize}
\end {enumerate}





\pagebreak




%-----------        ----------- 
%----------- Part 3 ----------- 
%-----------        ----------- 





\begin{enumerate}

\begin{flushleft}
{\large 3. Determine whether the function $\mathnormal{f}$ : $\mathbb{Z}$ × $\mathbb{Z}$ ! $\mathbb{Z}$ is onto if}\\
\end{flushleft}


%========= Problem A .===========

\large (a) $\mathnormal{f}$(m,n) =$m^2-n^2$\\

\begin{itemize}
\item \textbf{Solution}\\
\large - let $y \in $\mathbb{ Z }$\mid$\mathnormal{ f}$(m,n) = y$\\
\large - so we have $m^2-n^2=y$\\
\large - if n = 0 we get, $m^2 = y $\longrightarrow $ m = \textpm \sqrt{y}$ $\oplus$ $\mathbb{Z}$\\
\large - this tells us the preimage of y is nonexistent in $\mathbb{Z}$ × $\mathbb{Z}$\\
\large $\therefore$ $\mathnormal{f}$ is onto\\


\end {itemize}
\end {enumerate}

%========= Problem b .===========
\begin{enumerate}

\begin{flushleft}
\end{flushleft}


\large (b) $\mathnormal{f}$(m,n) = m + n + 1\\

\begin{itemize}
\item \textbf{Solution}\\
\large - let p $\in \mathbb{ Z }$ $\mid$ $\mathnormal{f}$(m,n) = p\\
\large - We can say that $p = m + n + 1$ \\
\large - Which implies that each of the elements of the given\\
\large co-domain set $\mathbb{ Z }$ has a preimage within the domain $\mathbb{Z}$ × $\mathbb{Z}$\\
\large $\therefore$ $\mathnormal{f}$ is onto\\


\end {itemize}
\end {enumerate}

%========= Problem c.===========
\begin{enumerate}

\begin{flushleft}
\end{flushleft}


\large (c) $\mathnormal{f}$(m,n) =$\lvert m  \rvert$ - $\lvert n \rvert$ \\

\begin{itemize}
\item \textbf{Solution}\\
\large let $y \in \mathbb{ Z }$ $\mid$ $\mathnormal{f}$(m,n) = y\\
\large - We can say that $y = \lvert m  \rvert$ - $\lvert n \rvert$\\
\large - Which illustrates that y has a preimage within $\mathbb{Z}$ × $\mathbb{Z}$\\
\large - We know that y is an arbitrary value, which means each\\ element in co-domain set $\mathbb{ Z}$ has a preimage in
$\mathbb{Z}$ × $\mathbb{Z}$\\
\large $\therefore$ $\mathnormal{f}$ is onto\\


\end {itemize}
\end {enumerate}


\pagebreak



%========= Problem d.===========
\begin{enumerate}

\begin{flushleft}
\end{flushleft}


\large (d) $\mathnormal{f}$(m,n) =$\lvert m  \rvert$ - $\lvert n \rvert$ \\

\begin{itemize}
\item \textbf{Solution}\\
\large - let $y \in \mathbb{ Z }$ be an element in codomain set in  $\mathbb{Z}$ \mind that $\mathnormal{f}$(m,n) = y\\
\large - which means that $m^2 - 4 = y$\\
\large $m^2 = y + 4$\\
\large $m = \textpm \sqrt{y+4}$ $\notin$ $\mathbb{Z}$ × $\mathbb{Z}$\\
\large - so $y$ has no preimage in the domain of $\mathbb{Z}$ × $\mathbb{Z}$\\
\large $\therefore$ $\mathnormal{f}$ is onto\\


\end {itemize}
\end {enumerate}





%-----------        ----------- 
%----------- Part 4 ----------- 
%-----------        ----------- 





\begin{enumerate}

\begin{flushleft}
{\large 3. Consider these functions from the set of students in a discrete mathematics class. State under what
conditions is the function one-to-one if it assigns to a student the following things below. Also state
how likely this is in practice.}\\
\end{flushleft}


%========= Problem A .===========

\large (a) Mobile phone number.\\

\begin{itemize}
\item \textbf{Solution}\\
\large If each student is assigned a unique phone number than the function is considered one-to-one.\\
\large so no two students must have the same number.
\large which is the case since not one person can have the same number, but numbers can be reused for new users\\


\end {itemize}
\end {enumerate}

%========= Problem b .===========
\begin{enumerate}
\begin{flushleft}
\end{flushleft}


\large (b) Student identification number.\\

\begin{itemize}
\item \textbf{Solution}\\
\large If each student has a unique Student ID number, than the function is one-to-one.\\
\large  Because this is used to identify an individual numbers must not and are not the same to avoid confusion\\


\end {itemize}
\end {enumerate}


%========= Problem c .===========
\begin{enumerate}
\begin{flushleft}
\end{flushleft}


\large (c) Final grade in the class.\\

\begin{itemize}
\item \textbf{Solution}\\
\large If each student has a unique final grade than the function is one-to-one\\
\large so no two students can have the same grade\\
\large this is rarely the case because there are only 4 grades to choose from and classrooms have many students\\

\end {itemize}
\end {enumerate}


%========= Problem d.===========
\begin{enumerate}
\begin{flushleft}
\end{flushleft}


\large (d) Home town.\\

\begin{itemize}
\item \textbf{Solution}\\
\large If each student is from a unique home town, than the function is one-to-one.\\
\large this is not the case at all because not everyone can be from seperate towns. \\

\end {itemize}
\end {enumerate}





%-----------        ----------- 
%----------- Part 5 ----------- 
%-----------        ----------- 





\begin{enumerate}

\begin{flushleft}
{\large 5. Determine whether each of these functions is a bijection from $\mathbb{R}$ to $\mathbb{R}$}\\
\end{flushleft}


%========= Problem A .===========

\large (a) $\mathnormal{f}$(x) = $-3x + 4$\\

\begin{itemize}
\item \textbf{Solution}\\
\large - let $x_1$ and $x_2 \in \mathbb{R}$ \mid $ x_1 \neq x_2$\\
\large - we let  $\mathnormal{f}$ $(x_1) = -3x_1+4$ and $\mathnormal{f}$ $(x_2) = -3x_2+4$\\
\large - $\mathnormal{f}(x_1) \neq \mathnormal{ f}(x_2)$\\
\large - $-3x_1 \neq -3x_2$\\
\large - Since $x_1 \neq x_2$ \\
\large $\therefore \mathnormal{f}(x)$ is one-to-one\\
\large - Now $\forall y \in \mathbb{R}$ and $\exists x = 4-y/3 \in \mathbb{R}$ \\
\large Such that $ \mathnormal{f}(x) = y$\\
\large - we know $\mathnormal{f}(4-y/3) = y$\\
\large - Then $ \mathnormal{f}(x)$ is onto\\
\large $\therefore \mathnormal{f}$ is a Bijection from $\mathbb{R}$ to $\mathbb{R}$\\


\end {itemize}
\end {enumerate}



%===================== Problem b .=============================
\begin{enumerate}
\begin{flushleft}
\end{flushleft}
\large (b) $\mathnormal{f}$(x) = $-3^2+7$\\

\begin{itemize}
\item \textbf{Solution}\\
\large - $\mathnormal{f}$(x) is not a one-to-one function\\
\large - If we take $\mathnormal{f}$(1) = 4 = $\mathnormal{f}$(-1)\\
\large - and 1 $\neq$-1 so it is not one-to-one\\
\large $\therefore \mathnormal{f}(x)$ is not a Bijection\\


\end {itemize}
\end {enumerate}



%===================== Problem c .=============================
\begin{enumerate}
\begin{flushleft}
\end{flushleft}

\large (c) $\mathnormal{f}$(x) = (x+1)/(x+2)\\

\begin{itemize}
\item \textbf{Solution}\\
\large - This is not a function because there is not \\
\large image of $x = -2 \in \mathbb{R}$\\
\large - so it cant be one-to-one nor onto\\
\large $\therefore \mathnormal{f} (x)$ is not a Bijection\\


\end {itemize}
\end {enumerate}


%===================== Problem d .=============================
\begin{enumerate}
\begin{flushleft}
\end{flushleft}


\large (c) $\mathnormal{f}(x) = x^3$\\

\begin{itemize}
\item \textbf{Solution}\\
\large - Let $\mathnormal{f}(x_1) \neq \mathnormal{f}(x_2)$\\
\large - Also let $x_1$ and $x_2$ $\in \mathbb{R}$\\
\large - We have $x_1^3 \neq x_2^3$\\
\large - Then $x_1 \neq x_2$\\
\large - So we can say $\mathnormal{f}(x)$ is one-to-one\\
\large - Now $\forall y \in \mathbb{R}$ and $\exists x = y^1/3 \in \mathbb{R}$ \\
\large - such that $\mathnormal{f}(x) = y$\\
\large - Then $\mathnormal{f}(x) = \mathnormal{f}(y^1/3)$ \longrightarrow $(y^1/3)^3$\\
\large - So $\mathnormal{f}(x) = y$ which makes it onto\\
\large $\therefore \mathnormal{f}(x)$ is a Bijection\\


\end {itemize}
\end {enumerate}







\end{document} 