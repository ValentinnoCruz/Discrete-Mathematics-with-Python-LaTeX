\documentclass[11pt]{article}
\documentclass[11pt]{article}
\documentclass{article}
\usepackage{libertine}
\usepackage{amssymb}
\usepackage{amsmath}
\usepackage[makeroom]{cancel}
\usepackage{indentfirst}

% To produce a letter size output. Otherwise will be A4 size.
\usepackage[letterpaper]{geometry}
% To produce a letter size output. Otherwise will be A4 size.
\usepackage[letterpaper]{geometry}

% For enumerated lists using letters: a. b. etc.
\usepackage{enumitem}

\topmargin -.5in
\textheight 9in
\oddsidemargin -.25in
\evensidemargin -.25in
\textwidth 7in
\usepackage{pifont}

\begin{document}





\begin{center}
    {\Large Valentinno Cruz\\
    Homework Assignment 5\\
    April. 29st, 2021\\}

\end{center}

%-----------                    ----------- 
%----------- Division Algorithm ----------- 
%-----------                    ----------- 

\begin{flushleft}
{\Large 1. The Division Algorithm}
\end{flushleft}



%_____________           _____________
%_____________ Problem 1 _____________ 
%_____________           _____________


\begin{flushleft}
{\large \hspace{.5cm}\textbf{1. What are the quotient and remainder when}\\
\end{flushleft}



%========= Problem a .===========


\begin{enumerate}

\begin{flushleft}
{\large a) 44 is divided by 8?}\\
\end{flushleft}





\begin{itemize}

\item \textbf{Solution}\\
\large Set 44 = q*8+r\\
\large here we can now start to solver for q and r\\
\large now we solve for q and r\\
\large $\therefore$ quotient (q) = 5 and remainder (r) = 4\\



\end {itemize}
\end {enumerate}


%========= Problem b .===========


\begin{enumerate}

\begin{flushleft}
{\large b) -1 is divided by 23?}\\
\end{flushleft}





\begin{itemize}

\item \textbf{Solution}\\
\large Set -1 =  q*23 + r \\
\large now we solve for quotient (q) and remainder (r)\\
\large $\therefore$ q = -1 and r = 22\\



\end {itemize}
\end {enumerate}



%========= Problem c .===========


\begin{enumerate}

\begin{flushleft}
{\large c) -2002 is divided by 87?}\\
\end{flushleft}





\begin{itemize}

\item \textbf{Solution}\\
\large Set -2002 = q*87 + r\\
\large now we solve for quotient (q) and remainder (r)\\
\large this gives us q = -24 and r = 86\\



\end {itemize}
\end {enumerate}



%========= Problem D .===========


\begin{enumerate}

\begin{flushleft}
{\large d) 0 is divided by 17?}\\
\end{flushleft}





\begin{itemize}

\item \textbf{Solution}\\
\large Set 0 = q*17 + r\\
\large now we solve for quotient (q) and remainder (r)\\
\large since the result is 0, this means both q & r = 0\\



\end {itemize}
\end {enumerate}


\pagebreak

%_____________           _____________
%_____________ Problem 2 _____________ 
%_____________           _____________


\begin{flushleft}
{\large \hspace{.5cm}\textbf{2. What time does a 24-hour clock read}\\
\end{flushleft}




%========= Problem a .===========


\begin{enumerate}

\begin{flushleft}
{\large a) 100 hours after it reads 2:00?}\\
\end{flushleft}





\begin{itemize}

\item \textbf{Solution}\\
\large Since it is 2:00 we can add it to 100hrs\\
\large this will give us 102hrs\\
\large now we divide 102 by number of hours in a day (24)\\
\large this give us a remainder of 6 which represents the time it will be after 100 hours\\
\large $\therefore$ 6:00\\



\end {itemize}
\end {enumerate}





%========= Problem b .===========


\begin{enumerate}

\begin{flushleft}
{\large b) 168 hours after it reads 19:00?}\\
\end{flushleft}





\begin{itemize}

\item \textbf{Solution}\\
\large Since it is 19:00 we add 19 + 168\\
\large this gives us 187\\
\large we divide 187 by 24 \\
\large this gives us a remainder of 19\\
\large $\therefore$ 19:00\\



\end {itemize}
\end {enumerate}







%_____________           _____________
%_____________ Problem 3 _____________ 
%_____________           _____________


\begin{flushleft}
{\large \hspace{.5cm}\textbf{2. Decide whether each of these integers is congruent to 3 modulo 7.}\\
\end{flushleft}




%========= Problem a .===========


\begin{enumerate}

\begin{flushleft}
{\large a) 37}\\
\end{flushleft}





\begin{itemize}

\item \textbf{Solution}\\
\large when we take 37 mod 7 we get a remainder of 2\\
\large 2 $\neq$ 3\\
\large $\therefore$ not congrudent\\



\end {itemize}
\end {enumerate}


%========= Problem b.===========


\begin{enumerate}

\begin{flushleft}
{\large a) -17}\\
\end{flushleft}





\begin{itemize}

\item \textbf{Solution}\\
\large when we take -17 mod 7 we get a remainder of 4\\
\large 4 $\neq$3\\
\large $\therefore$ not congruent\\


\end {itemize}
\end {enumerate}


\pagebreak

%-----------                    ----------- 
%----------- Modular Arithmetic ----------- 
%-----------                    ----------- 

\begin{flushleft}
{\Large 2. Modular Arithmetic.}
\end{flushleft}



%_____________           _____________
%_____________ Problem 1 _____________ 
%_____________           _____________


\begin{flushleft}
{\large \hspace{.5cm}\textbf{1. Complete the following operations modulo m where m = 13.}\\
\end{flushleft}



%========= Problem a .===========


\begin{enumerate}

\begin{flushleft}
{\large a) 4$+_m$11}\\
\end{flushleft}





\begin{itemize}

\item \textbf{Solution}\\
\large we know 4 + 11 = 15\\
\large then 15 = 2*4 + 7\\
\large and then 15 mod 4 = 7\\
\large $\therefore$ 4$+_m$11 = 7\\



\end {itemize}
\end {enumerate}


%========= Problem b .===========


\begin{enumerate}

\begin{flushleft}
{\large a) 4$*_m$11}\\
\end{flushleft}





\begin{itemize}

\item \textbf{Solution}\\
\large we know 4 * 11 = 44\\
\large then 44 = 3*13 + 5\\
\large and then 44 mod 13 = 5\\
\large $\therefore$ 4$*_m$11 = 5\\



\end {itemize}
\end {enumerate}


\end{document} 